\label[chapter_2]
\chap Identification of decision-making processes realized by humans in Starcraft
In this chapter we want to introduce real-time strategy (RTS) games, give an overview of decision-making processes of player in those games, especially in the domain of StarCraft:~Brood War, and show that mastering any RTS game is nothing trivial, even for a human.

\label[complexity]
\sec Real-time strategy games and their complexity
RTS is one of the sub-genre of strategy games. In those kinds of games, the player usually needs to build economy (collect resources and construct buildings) and military power (by training and upgrading units in buildings for gathered resources) to defeat his opponents (by destroying his army or economy). The \uv{real-time} gives RTS games other dimensions compare to a classic game of Chess. Each player has a small time frame to decide the next move in an environment where players’ actions are simultaneous as players can issue play command in same time. Most of the actions are not instantaneous; it takes some time to complete them to see the result. On top of that, the RTS games are partially observable as players do not have full perception of the state of the affair in the world. To this situation is referred as fog-of-war because the player can not see unexplored parts of the maps and do not know situations in parts of the map where he does not have his units. Moreover, those games are usually non-deterministic as actions may not succeed given their chance of failure, and most importantly the complexity of the state and action space is enormous.

In~\cite[BenG.Weber.2012] author gives decision space complexity (set of possible actions which can be executed at a particular moment) estimation of RTS trough StarCraft as follows:
\medskip
\centerline{\eqmark~$\mathcal{O}\left (  (W\cdot A\cdot P)(T\cdot D\cdot S)+B\cdot (R+C)\right )$}
\medskip
\begitems
* W - number of workers
* A -number of the type of worker assignments
* P -average number of workplaces
* T -number of troops
* D -number of movement directions
* S -number of troop stances (Attack, Move, Hold)
* B -number of buildings
* R -average number of research options at buildings
* C -average number of unit types at buildings
\enditems
\medskip
Given an extremely simplified scenario of the game on 256x256 tile map in SC: BW with 50 workers results in 1 000 000 000 possible actions which is orders of magnitude higher comparing it to the complexity of Chess. Same goes for state complexity. For example, Chess is estimated to be around $10^{50}$ and Go around $10^{170}$. However, StarCraft scenario on the typical map is believed to be many orders of magnitude larger. More detailed discussion of StarCraft complexity can be found in~\cite[Ontanon2013].

\medskip \clabel[alphago]{AlphaGo plays Lee Sedol in 2016}
\picw=15cm \cinspic alphago.jpg
\caption/f AlphaGo plays Lee Sedol in 2016 (from:~\cite[wEdtRVuUf1ducyzE]).
\medskip

\label[decisions]
\sec Decision-making processes in RTS games
Due to the complex nature of problem playing RTS games, the most common approach is by decomposition of the problem into a collection of subproblems which can be solved independently. Conventional subdivision (not the exclusive one) is according to~\cite[Ontanon2013] as follows:
\begitems 
*The {\bf Strategy} is the most abstract level of game comprehension. It corresponds to the high-level decision-making process and concerns all units and building as well as properties of the environment. Finding successful strategy against given opponent is key to defeating him.
*{\bf Tactics} is a way how to realize strategy. It focuses on groups of units and implies theirs positioning, movements, timings and so forth.
*{\bf Reactive control} is the implementation of tactics concerning particular unit. It involves moving, targeting, fleeing and so on.
*{\bf Terrain analysis} is part of environment analysis (map specifically), the primary goal is for example to identify strategic locations, resources, and distances. This knowledge is then employed in other processes.
*{\bf Intelligence gathering} that corresponds to information collection due to the partial observability of environment to gain intelligence on the opponent.
\enditems

Levels of abstraction described above and their relation to uncertainty coming from partial observability and not knowing specific intentions of the opponent, timing corresponding to a duration of behavior switching, and spatial and temporal reasoning are on figure~\ref[abstraction].

\medskip \clabel[abstraction]{RTS AI levels of abstraction and theirs properties}
\picw=10cm \cinspic abstraction.png
\caption/f RTS AI levels of abstraction and theirs properties (from:~\cite[Ontanon2013]).
\medskip

Churchill in~\cite[Churchill2016] gives a deeper elaboration of Strategy, Tactics and Reactive Control description and their subtasks as can be seen on figure~\ref[subproblems]. On this figure can also be seen information flow hierarchy between those subtasks similar to the military command structure. 

\medskip \clabel[subproblems]{The main sub-problems in RTS AI research categorized by their approximate time scope and level of abstraction}
\picw=10cm \cinspic subproblems.png
\caption/f The main sub-problems in RTS AI research categorized by their
approximate time scope and level of abstraction (from:~\cite[Churchill2016]).
\medskip

Subtasks for {\bf Strategy} are as follows:
\begitems
* {\bf Knowledge and learning} which are vital for playing the game (rules, unit properties, openings, knowledge about the opponent and so long). This knowledge can be further extended by playing matches and gathering additional information about the game itself and opponents.
* {\bf Opponent modeling and prediction} is useful to the player because the environment is only partially observable so one can not be sure what is his opponent doing. In this case, modeling and prediction come in handy as its enables player to exploit perceived weaknesses.
* {\bf Strategic Stance} corresponding to the ability to balance between aggression and economic expansion. The choice of particular stance influences things such as army composition and attack timing.
* {\bf Army composition} is decided by Strategic Stance, Opponent modeling and prediction, and by the current situation.
* {\bf Build-Order Planning} goal is realizing decided army composition by gathering resources and building infrastructure to have resources, capacity and meet requirements to train desired units.
\enditems

As for {\bf Tactics} subtasks are following:
\begitems
* {\bf Scouting} is a way to gather intel on the opponent as areas of RTS maps are typically covered by fog-of-war which unable vision of other sectors and enemies expect those in the direct vicinity of friendly units. Sometimes is possible to use technology or ability to uncover part of the map. Scouting is a vital part of the game as it provides information to the player to adjust his play and make right decisions.
* {\bf Combat Timing and Positioning} are crucial in RTS games as it involves decision where and when to attack the opponent. For example attacking opponent’s new expansion when it still under construction can help the player get an edge in a match.
* {\bf Building Placement} plays most important role in the game, especially in the beginning and is very influenced by map, opponent, strategy and so. For example, one can imagine situation by placement of additional defensive buildings player can discourage the enemy from attack, or at least delay it.
\enditems

For {\bf Reactive Contro}l Churchill in~\cite[Churchill2016] list following subtasks:
\begitems
* {\bf Unit Micro} is mostly referred to in combat situations as it dictates individual units behavior. Decision processes here are particularly hard as one needs to manage many units and their actions simultaneously. 
* {\bf Multi-Agent Pathfinding and Terrain Analysis} are an integral part of RTS games. Pathfinding is related to Unit Micro and involves pathfinding and other more complex optimizations to avoid getting damage and so. Terrain analysis gives player not just necessary inputs for pathfinding algorithms but high-level info about properties of the map such as resources location.
\enditems

Despite the decomposition, most of the task remains quite complex as is illustrated in~\cite[Viglietta2014] where many games (even StarCraft) are analyzed and from the discussion of researchers and authors of SC:~BW agents in~\cite[BAP1jrCPSHw0ki9n]. On top of that for many of the tasks, the optimal solution does not exist which is shown in \ref[chapter_3].

\sec StarCraft specifics
StarCraft and its expansion StarCraft: Brood War was released in 1998 by Blizzard Entertainment and became tremendously popular. To win the game, mechanics are similar to traditional RTS games. The player needs to collect resources (in this case mineral and gas) to be able to construct buildings, train units and unlock upgrades for them. Resources further can be spent on defensive buildings to help units to defend strategic points of a map given the specific situation. Player uses units, not just for defense; he distributed them among other tasks such as scouting, attack, maneuvering them when they meet the enemy, and various \uv{mind games} to confuse the opponent.

\medskip \clabel[starcraft]{StarCraft: BroodWar in-game screen}
\picw=15cm \cinspic starcraft.jpg
\caption/f StarCraft: BroodWar in-game screen.
\medskip
 
The game is set in a science-fiction universe where the player takes the role of commander of an army of one of the three races:
\begitems
* Zergs as an insectoid alien race with cheap and weak units which can be produced fast giving the player ability to overwhelm opponent simply by numbers.
* Protoss is also alien race similar to humans. Units are the exact opposite of Zergs’. They have expensive production regarding costs and manufacturing times. Playing this race means to concentrate more on quality over quantity.
* Terrans represents humans in this world with units balanced between Zergs and Protoss. 
\enditems

Despite the fact that each race contains 30-35 unique types of buildings and units (most of them has own special abilities) is extremely well balanced. The typical game takes place on the map of dimension between 64x64 to 256x256 build tiles (32x32 squares of pixels). The game does not allow each player to control more than 200 units at the time, but the player can have an unlimited number of buildings. All of this makes StarCraft challenging on the one hand, but extremely fun to play (watch) on the other. Thanks to this StarCraft became famous eSport matter with contestants earning a considerable amount of money~\cite[Y4gR999wLBmcbWSC].

In comparison to~\ref[decisions] when humans playing StarCraft they typically abstract decision making in this manner:
\begitems
* {\bf Micromanagement} (Micro) corresponds to Reactive Control and partially to Tactics described in {previous section}. It involves the ability to control units individually. Good micro players are more likely to keep their units alive longer.
* {\bf Macromanagement} (Macro) corresponds almost to everything except Reactive Control and Tactics (except the part from micro). It represents the capacity to produce right units and expand at appropriate times to ensure that production of units is flowing. A player who is macro-oriented has a usually larger 
\enditems

This chapter is especially vital for our bot development as we had very sparse knowledge about StarCraft:~BroodWar game. In~\ref[chapter_7] we present implementation of our bot based on knowledge from this chapter.

